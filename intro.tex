\chapter{Introduction}
\label{chap:intro}

With CPU speeds stagnating, the standard approach for increasing performance
is by using more CPUs rather than faster CPUs.  The cost of scaling
the number of cores on a multi-core CPU is quite prohibitive and
renders the model of vertical scaling both inflexible and inelastic
for a range of real-world applications. Horizontally scaling the
computation across a range of commodity compute nodes is much more
cost effective, scales more incrementally and is becoming increasingly
popular for programming compute intensive distributed applications. We
refer to this approach of renting a cluster of nodes with on-demand
elastic scaling of resources and cost as ``cloud computing''. Cloud
Haskell {cite paper here} is a framework for writing distributed applications in
Haskell.

\section{Challenges in a distributed system}

Developing distributed programs which can scale horizontally across a
large number of clusters presents some unique challenges:

\begin{itemize}
\item When programming the cluster as a whole, there is a need to
  coordinate various processes running on heterogeneous systems. Most
  programming languages do not directly address the problem of
  distributed concurrency. The dominant model of concurrency in most
  mainstream programming languages is usually the shared-memory
  variety. It relies on the concept of multiple threads modifying
  shared mutable data. This model is not very useful in a distributed
  model. HdH {cite} tries to replicate the model of shared memory concurrency
  in a distributed system by using the concept of distributed
  MVars. This model is not very successful because the cost of moving
  data across in a distributed system becomes a dominant factor. By
  making message passing explicit, Cloud Haskell exposes the cost of
  message passing to the programmer.
\item The problem of fault tolerance becomes non-trivial in a
  distributed system. When a distributed program is running across
  hundreds of thousands of nodes, some of the nodes will fail at any
  given moment of time. A failure of a node should not require
  restarting the whole calculation , or it might never finish. The
  programmer needs tools to detect and respond to failures as part of
  the programming model. Cloud Haskell allows monitoring of processes
  and exposes primitives to handle node failures.
\end{itemize}

\section{The Problem of Hot Code Reloading}

\subsection{The Problem}

Current cloud haskell systems cannot be easily upgraded from one
verison of the code to the next. There is no support in the tool to
enable safe upgrades. Moreover, if there are multiple versions running
concurrently, processes having incompatible types won't communicate
and the current implementation fails silently. This makes debugging
very hard since the programmer cannot figure out if no error is
generated for incompatible versions. Ad hoc update mechanisms like
using external tools for updating a cluster are neither safe nor
efficient and simply restart the whole system.

In this thesis, we work on the problem of dynamically updating a
running cloud haskell system with zero downtime.

\subsection{The Motivation}

\begin{itemize}
\item Zero downtime is a fairly common requirement in large scale
  distributed systems running critical business logic. This is
  especially true in financial transaction systems, telephone
  switches, airline traffic control systems and other mission
  critical systems.
\item Hot code reloading is less expensive than using redundant
  hardware for managing upgrades. Loss of web service during
  maintenance is no more acceptable and leads to lost revenues. For
  example, Visa makes use of 21 main frames to run its
  fifty-million-line transaction processing system. It is selectively
  able to take machines down and upgrade them by preserving relevant
  state in other online systems and complex state migration. This
  approach is expensive as well as increases the complexity of
  deploying updates.
\item It helps in rapid prototyping and increases developer
  productivity by reducing the length of an iteration cycle. The
  ability to modify a running program and update it on the fly saves a
  lot of time which would be otherwise wasted in restarting a program
  and rebuilding all the relevant state. Moreover, the real time
  feedback available to the programmer when he changes the program is
  very valuable in supporting interactive programming.
\item Language level updating facility is more reliable than using
  external tools to safely update a software. Manually distributing
  the updated version using tools like scp, puts the burden of
  responsibility on the programmer to make sure that all nodes are
  running the same node. This can be quite error-prone because Cloud
  Haskell would silently fail during communication between processes
  of incompatible types.
\end{itemize}

\subsection{Example of Hot Code Reloading in Erlang}

Erlang supports language level dynamic software updating. A process in
erlang can update into the new version by making an external call to
its module.

We demonstrate an example of hot code reloading in erlang by the help
of a counter process. It can receive a message to increment the
running counter, or a message to send the current counter value, it
can also receive a message to update itself.

\begin{program}
\caption[An example of hot code loading]
{Hot code loading in erlang : version 1}
\label{fig:erlang-v1}

\begin{minted}{erlang}
  %% A process whose only job is to keep a counter.
  %% First version
  -module(counter).
  -export([start/0, codeswitch/1]).

  start() -> loop(0).

  loop(Sum) ->
    receive
       {increment, Count} ->
          loop(Sum+Count);
       {counter, Pid} ->
          Pid ! {counter, Sum},
          loop(Sum);
       update ->
          ?MODULE:codeswitch(Sum)
          % Force the use of 'codeswitch/1' from the latest MODULE version
    end.

  codeswitch(Sum) -> loop(Sum).
\end{minted}
\end{program}

In version 2, we add the possibility to reset the counter to 0.

\begin{program}
\caption[An example of hot code loading]
{Hot code loading in erlang : version 2}
\label{fig:erlang-v2}

\begin{minted}{erlang}
 %% Second version
  -module(counter).
  -export([start/0, codeswitch/1]).

  start() -> loop(0).

  loop(Sum) ->
    receive
       {increment, Count} ->
          loop(Sum+Count);
       reset ->
          loop(0);
       {counter, Pid} ->
          Pid ! {counter, Sum},
          loop(Sum);
       update ->
          ?MODULE:codeswitch(Sum)
    end.

  codeswitch(Sum) -> loop(Sum).
\end{minted}
\end{program}

On receiving a ``update'' message, loop will execute an external call
to codeswitch. If there is a new version of ``counter'' module in
memory, the its codeswitch function will be called with the update
state. In our example, we pass the same state to the new version.

The goal of hot code reloading in Cloud Haskell is to have similar
style of code upgrade facility as Erlang.

\section{Performance issues in Cloud Haskell}

While working on the problem of hot code reloading, we came across a
number of performance issues in the current implementation of Cloud
Haskell. Compared to Erlang which has been extensively used in large
scale production systems and extensively improved since last twenty
years, Cloud Haskell was developed only two years back. It has not
been extensively used in production nor any extensive bench-marking
been done with other systems like Erlang. Performance is a major
concern when developing large scale distributed systems which cost a
lot of resources and money.

\section{Our Contributions}

The contributions made in this thesis are:
\begin{itemize}
\item A prototype implementation for upgrading a single cloud haskell
  instance with possibility of intermittent message loss during the
  update.
\item A discussion on other potential approaches to hot code reloading
  in Cloud Haskell like systems.
\item An exhaustive evaluation of the trade-offs in the different
  approaches.
\item Investigation of some performance issues in the current release
  of Cloud Haskell specifically related to the performance of garbage
  collection.
\end{itemize}

The rest of the thesis is organized as follows. Chapter 2 gives a
brief tour of Cloud Haskell and the design decisions taken which set
the background for understanding the rest of the thesis. Chapter 3
highlights the major related work in the field of Dynamic Software
Updating. Chapter 4 describes our approach to hot code reloading in
Cloud Haskell. Chapter 5 briefly talks about the performance issues we
found in Cloud Haskell.