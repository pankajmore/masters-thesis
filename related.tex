\chapter{Related work}

The overall field concerning hot code reloading in software is called
Dynamic Software Updating (DSU). DSU is a field of research focused on
safely upgrading programs while they are running {cite}. Although
there is no existing research on DSU in the area of distributed
functional programming, there is lot of work in the field of
imperative and object oriented programming languages. We will briefly
describe the main ideas and concepts found in the literature. For a
detailed survey of DSU systems, {cite} is a very good resource.

\section{Formal specification of DSU}

Any running program can be thought of a tuple (\delta, P) . \delta is the
current program state and P is the current program code. DSU systems
transfer a running program (\delta, P) to (\delta', P'). The state
must be transformed into a representation P' expects. This requires a
state transformer function. Thus, DSU transforms (\delta, P) to
(S(\delta), P). An update is considered valid if and only if the
running program (S(\delta), P') can be reduced to a point tuple
(\delta, P') that is reachable from the starting point of the new
version of the program, ($\delta_{init}$, P').  For a formal
description of this ``validity'' , please refer to {cite}.

Although this is one formal definition for DSU, there is no consensus
on the standard definition of DSU applicable in all cases. {cite}
presents a list of definitions and requirements according to different
authors.

\section{Related ideas and techniques}

In this section, we briefly discuss the most relevant concepts and
techniques commonly used in the field of DSU systems.

\subsection{Quiescence}


\subsection{Binary Code Rewriting}

Some authors mostly in Java and C propose rewriting of the binary code of the programs in memory. Fraby was one of the fir 
\subsection{Proxies, Intermediaries and Indirection Levels}

\subsection{Intrusion and Cooperation}

\subsection{State Transfer}

\subsection{Transformation Functions}

\subsection{Using underlying facilities}

\subsection{Version Coexistence}


\section{DSU and Functional Programming}

\subsection{Haskell}

dons thesis
safecopy

\subsection{Erlang}


erlang's unspecified upgrade behaviour
Erlang requires no safety guarantees on updates cite.
haskell reloading facility by don;s thesis
